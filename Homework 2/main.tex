\documentclass[twocolumn]{article}


\author{Matthias Maile}
\date{March 2022}
\title{Homework 2}

\usepackage{amsmath}
\usepackage{graphicx}
\usepackage{float}

% page setup
\usepackage[a4paper, margin=2.5cm]{geometry}

\usepackage{multicol} % multicolumn text

% references
\usepackage{hyperref}

\usepackage{unicode-math}


\begin{document}
\maketitle
\tableofcontents

\section{Differential Equations}
In this part the ordinary differential equation
\[
  y^\prime = t ( 1-y^2)
\]
has been solved numerically. It can be solved exactly by seperating the variables or by 
\href{https://www.wolframalpha.com/input?i=y%27+%3D+t+*+%281-y%5E2%29%2C+y%280%29+%3D+-0.5}{not living under a rock}. With the
later you get the result 
\[
  y(t) = \frac{e^{t^2} - 3}{3 + e^{t^2}}.
\]
The error of each numerical method has been plotted in \autoref{fig:error-ode}. The maximum error for many different
$dt$ values is plotted in \autoref{fig:error2-ode}. I compared the Runge-Kutta and Euler method as given and used the
\href{https://docs.scipy.org/doc/scipy/reference/generated/scipy.integrate.odeint.html}{scipy.integrat.odeint}
method which uses something from Fortran under the hood.
\begin{figure}
	\centering
  \includegraphics[width=0.5\textwidth]{build/Ex1-1.pdf}
	\caption{Error of Eulers method, Runge Kutta and scipy.integrate.odeint plotted for $dt = 0.01$ and $dt = 0.005$.}
  \label{fig:error-ode}
\end{figure}
\begin{figure}
	\centering
  \includegraphics[width=0.5\textwidth]{build/Ex1-2.pdf}
	\caption{Maximum error of the different integration methods for different time steps $dt$.}
  \label{fig:error2-ode}
\end{figure}

\section{Solution of equation and basin of attraction}
\subsection{Newton`s method to solve an equation}
In this part the equation
\begin{equation}
  0 = \frac12 + \frac14 x^2 - x \sin x - \frac12 \cos 2x
  \label{eqn:newton}
\end{equation}
will be solved using Newton`s method. The result is shown in \autoref{fig:smallx0} for the initial value $x_0 =
\frac{\pi}{2}$. It is also shown for $x_0 = 5 \pi$ and $x_0 = 10 \pi$ in \autoref{fig:mediumx0} and
\autoref{fig:largex0}. The point where the right side in \autoref{eqn:newton} is zero has a vanishing gradient, thus Newton`s method is not perfect
in its default form, though it still converged here.
\begin{figure}
	\centering
  \includegraphics[width=0.5\textwidth]{build/Ex2-1.pdf}
  \caption{Newton`s method to solve \autoref{eqn:newton} with $x_0 = \frac{\pi}{2}$.}
  \label{fig:smallx0}
\end{figure}
\begin{figure}
	\centering
  \includegraphics[width=0.5\textwidth]{build/Ex2-2.pdf}
  \caption{Newton`s method to solve \autoref{eqn:newton} with $x_0 = 5\pi$.}
  \label{fig:mediumx0}
\end{figure}
\begin{figure}
	\centering
  \includegraphics[width=0.5\textwidth]{build/Ex2-3.pdf}
  \caption{Newton`s method to solve \autoref{eqn:newton} with $x_0 = 10\pi$.}
  \label{fig:largex0}
\end{figure}

\section{Roessler Attractor}
In this part we look at the Roessler attractor. The attractor itself is plotted for $c=5.7$, $c=6$and $c=14$ in
\autoref{fig:smallc}, \autoref{fig:mediumc} and \autoref{fig:largec}. To look at the periodic behaior in a more precise
way the the exercise suggested to look at $y(t)$. The Time evolution and its peak found with scipy are plotted in
\autoref{fig:smallc=yt}, \autoref{fig:mediumc-yt} and \autoref{fig:largec-yt}.
\begin{figure}
	\centering
  \includegraphics[width=0.5\textwidth]{build/Exercise3-5.7-3d.pdf}
  \caption{Roessler attractor for $c = 5.7$.}
  \label{fig:smallc}
\end{figure}
\begin{figure}
	\centering
  \includegraphics[width=0.5\textwidth]{build/Exercise3-6-3d.pdf}
  \caption{Roessler attractor for $c = 6$.}
  \label{fig:mediumc}
\end{figure}
\begin{figure}
	\centering
  \includegraphics[width=0.5\textwidth]{build/Exercise3-14-3d.pdf}
  \caption{Roessler attractor for $c = 14$.}
  \label{fig:largec}
\end{figure}

\begin{figure}
	\centering
  \includegraphics[width=0.5\textwidth]{build/Exercise3-5.7-2d-t-y.pdf}
  \caption{Peaks of $y$ for $c = 5.7$.}
  \label{fig:smallc-yt}
\end{figure}
\begin{figure}
	\centering
  \includegraphics[width=0.5\textwidth]{build/Exercise3-6-2d-t-y.pdf}
  \caption{Peaks of $y$ for $c = 6$.}
  \label{fig:mediumc-yt}
\end{figure}
\begin{figure}
	\centering
  \includegraphics[width=0.5\textwidth]{build/Exercise3-14-2d-t-y.pdf}
  \caption{Peaks of $y$ for $c = 14$.}
  \label{fig:largec-yt}
\end{figure}
\section{Diffusion limited aggregation}

\end{document}
