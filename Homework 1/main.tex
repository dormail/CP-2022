\documentclass{article}


\author{Matthias Maile}
\date{March 2022}
\title{Homework 1}

\usepackage{amsmath}
\usepackage{graphicx}
\usepackage{float}

% page setup
\usepackage[a4paper, margin=2.5cm]{geometry}

% references
\usepackage{hyperref}

\begin{document}
\maketitle
\tableofcontents

\section{Round-off Errors And Computer Arithmetic}
\label{sec:Round-off errors and Computer Arithmetic}
Here we are looking at the two formulae
\[
  e^{-5} \approx \sum_{i=0}^n \frac{(-5)^i}{i!}
  \qquad
  e^{-5} = \frac{1}{e^5} \approx \sum_{i=0}^n \frac{1}{\frac{5^i}{i!}}.
\]
As it can be seen in \autoref{fig:abs_error}, Formula 2 is converges much faster 
to the exact value. This is because the sum in Formula 1 is not uniform and has 
large jumps for small $i$.


\begin{figure}[H]
	\centering
	\includegraphics{build/Ex1.pdf}
	\caption{Plot of the absolute error for different $n$.}
	\label{fig:abs_error}
\end{figure}

\section{Algorithms And Convergence}
\label{sec:Algorithms and Convergence}

\section{Generating Random Number With Power Law Distribution}
\label{sec:Generating Random Number With Power Law Distribution}

\section{Trajectory Of A Random Walk}
\label{sec:Trajectory Of A Random Walk}


\end{document}
