\documentclass{article}


\author{Matthias Maile}
\date{March 2022}
\title{Homework 1}

\usepackage{amsmath}
\usepackage{graphicx}
\usepackage{float}

% page setup
\usepackage[a4paper, margin=2.5cm]{geometry}

% references
\usepackage{hyperref}

\begin{document}
\maketitle
\tableofcontents

\section{Round-off Errors And Computer Arithmetic}
\label{sec:Round-off errors and Computer Arithmetic}
Here we are looking at the two formulae
\[
  e^{-5} \approx \sum_{i=0}^n \frac{(-5)^i}{i!}
  \qquad
  e^{-5} = \frac{1}{e^5} \approx \sum_{i=0}^n \frac{1}{\frac{5^i}{i!}}.
\]
As it can be seen in \autoref{fig:abs_error1}, Formula 2 is converges much faster 
to the exact value. This is because the sum in Formula 1 is not uniform and has 
large jumps for small $i$.


\begin{figure}[H]
	\centering
	\includegraphics{build/Ex1.pdf}
	\caption{Plot of the absolute error for different $n$.}
	\label{fig:abs_error1}
\end{figure}

\section{Algorithms And Convergence}
\label{sec:Algorithms and Convergence}
The absolute error is plotted in \autoref{fig:abs_error2}. To calculate the degree of each formula, so 
$p$ such that
\[
  \vert \gamma_n - \gamma_inf \vert = K \frac{1}{n^p},
\]
I utilized the curve\_fit from the scipy.optimize library. I defined a function
\[
  f(n, K, p) = \frac{K}{n^p}
\]
so that the curve\_fit method could find optimal $K$ and $p$. This gave the results
\begin{align}
  K_1 &= 0.498 & p_1 &= 0.999, \\
  K_2 &= 0.697 \pm 0.089 & p_2 &= 0.74 \pm 0.023.
\end{align}
So we can tell that the first formula is of order $p\approx1$ and formula 2 $p\approx \frac{3}{4}$.

\begin{figure}[H]
	\centering
	\includegraphics{build/Ex2.pdf}
	\caption{Plot of the absolute error and the fitted curve for different $n$.}
	\label{fig:abs_error2}
\end{figure}


\section{Generating Random Number With Power Law Distribution}
\label{sec:Generating Random Number With Power Law Distribution}

\section{Trajectory Of A Random Walk}
\label{sec:Trajectory Of A Random Walk}

\subsection{Normal Walker}
\label{sec:Normal Walker}
In this part we take a look at the normal walker, i.e. a walker that can return.
In \autoref{fig:dist-returning} the trace for eight returning walkers is plotted. 
\begin{figure}[H]
	\centering
	\includegraphics{build/many_walks.pdf}
	\caption{Eight random walker`s trace plotted for 1000 steps.}
  \label{fig:dist-returning}
\end{figure}


\end{document}
