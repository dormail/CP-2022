\documentclass{article}


\author{Matthias Maile}
\date{March 2022}
\title{Homework 1}

\usepackage{amsmath}
\usepackage{graphicx}
\usepackage{float}

% page setup
\usepackage[a4paper, margin=2.5cm]{geometry}

\usepackage{multicol} % multicolumn text

% references
\usepackage{hyperref}

\usepackage{unicode-math}


\begin{document}
\maketitle
\tableofcontents

\section{Round-off Errors And Computer Arithmetic}
\label{sec:Round-off errors and Computer Arithmetic}
Here we are looking at the two formulae
\[
  e^{-5} \approx \sum_{i=0}^n \frac{(-5)^i}{i!}
  \qquad
  e^{-5} = \frac{1}{e^5} \approx \sum_{i=0}^n \frac{1}{\frac{5^i}{i!}}.
\]
As it can be seen in \autoref{fig:abs_error1}, Formula 2 is converges much faster 
to the exact value. This is because the sum in Formula 1 is not uniform and has 
large jumps for small $i$.

\begin{figure}[H]
	\centering
	\includegraphics{build/Ex1.pdf}
	\caption{Plot of the absolute error for different $n$.}
	\label{fig:abs_error1}
\end{figure}

\section{Algorithms And Convergence}
\label{sec:Algorithms and Convergence}
The absolute error is plotted in \autoref{fig:abs_error2}. To calculate the degree of each formula, so 
$p$ such that
\[
  \vert \gamma_n - \gamma_inf \vert = K \frac{1}{n^p},
\]
I utilized the curve\_fit from the scipy.optimize library. I defined a function
\[
  f(n, K, p) = \frac{K}{n^p}
\]
so that the curve\_fit method could find optimal $K$ and $p$. This gave the results
\begin{align}
  K_1 &= 0.498 & p_1 &= 0.999, \\
  K_2 &= 0.697 \pm 0.089 & p_2 &= 0.74 \pm 0.023.
\end{align}
So we can tell that the first formula is of order $p\approx1$ and formula 2 $p\approx \frac{3}{4}$.

\begin{figure}[H]
	\centering
	\includegraphics{build/Ex2.pdf}
	\caption{Plot of the absolute error and the fitted curve for different $n$.}
	\label{fig:abs_error2}
\end{figure}


\section{Generating Random Number With Power Law Distribution}
\label{sec:Generating Random Number With Power Law Distribution}
This part consists of two parts: Firstly, we have to derive a function $f$ such that $y = f(x)$ follows
a power law $P(y) = A \cdot y^{-\alpha}, y \in [2, \inf\}$ when $x$ is evenly distrubuted along $(0,1]$. 
In the second part we will implement it in python and test the generated distribution.

\subsection{Deriving $f(x)$}
For $f(x)$ we invert the comulative distribution function (CDF) from $y$. Firstly, we will simplify $P(y)$ 
by finding $A(\alpha$ (they are not independent because of the normalization condition):
\begin{align*}
  \int_{-\infty}^{\infty} P(y) \, \symup{d}y 
  &= \int_{2}^{\infty} P(y) \, \symup{d}y \\
  &= \int_{2}^{\infty} A y^{-\alpha} \, \symup{d}y \\
  &= \frac{-A}{-\alpha+1} \left. y^{-\alpha+1} \right\vert_2^\infty \\
  &= \frac{A}{\alpha - 1} 2^{1-\alpha} \overset{!}{=} 1 \\
  \Leftrightarrow
  A &= (\alpha - 1) 2^{\alpha - 1}
\end{align*}
Now we are ready to compute the CDF of $P(y)$:
\begin{align*}
  \symup{CDF}(y) &= \int_{-\infty}^{y} P(y^\prime) \, \symup{d}y^\prime  \\
                 &= \int_{2}^{y} (\alpha - 1) 2^{\alpha -1} y^{\prime-\alpha} \, \symup{d}y^\prime \\
                 &= (\alpha-1) 2^{\alpha-1} \cdot \frac{1}{-\alpha+1} \left. y^{\prime -\alpha +1} \right\vert_2^y \\
                 &= -2^{\alpha-1} \left[ y^{-\alpha+1} - 2^{-\alpha +1} \right] \\
                 &= 1 - \left( \frac{y}{2} \right)^{1-\alpha}.
\end{align*}
Finally we can invert it to find $y(x)$:
\begin{align*}
  \symup{CDF}(y) &= x \\
  x &= 1 - \left( \frac{y}{2} \right)^{1-\alpha} \\
  \Leftrightarrow 
  1 - x &= \left( \frac{y}{2} \right)^{1-\alpha} \\
  \Leftrightarrow
  2 \cdot \left(1 - x\right)^{\frac{1}{1-\alpha}} = y = f(x).
\end{align*}

\subsection{Implementation and Tests}
\label{sec:implementation}
To test the transformation 
\[
  f(x) = 2 \cdot \left(1 - x\right)^{\frac{1}{1-\alpha}}
\]
we will generate $10000$ samples of $x$ with the numpy.random.rand function and apply $f$ to all of them. For
$\alpha$ we chose the value 5.
The matplotlib.pyplot.hist function will sort those into 50 bins. 

To test the power-law-like behavior we will proceed like in \autoref{sec:Algorithms and Convergence} with 
the curve\_fit from the scipy.optimize library. Since we have intervals, we first compute the middle point
of each interval and than fit the function
\[
  p(x, A, \alpha) = A \cdot x^{-\alpha}
\]
to the data pairs (midpoint, occurences). This gives us a fit parameter 
\[
  \alpha = 5.24 \pm 0.017,
\]
so a deviation of about $4.8\%$ from the true $\alpha = 5$. The data histogram and the curve fit are plotted in 
\autoref{fig:ex3}.

As suggested in the exercise, there is also a log-log-plot. One visible feature of a power law is the linear-like
slope in this plot which should be $\approx \alpha$. This can be either confirmed by looking at 
\autoref{fig:ex3-log} or taking our measured $\alpha$, which corresponds to the actual log-log-slope.
\begin{figure}[H]
	\centering
	\includegraphics{build/Ex3.pdf}
	\caption{Generated powerlaw distribution and measured exponent for $\alpha = 5$, $10000$ samples.}
  \label{fig:ex3}
\end{figure}
\begin{figure}[H]
	\centering
	\includegraphics{build/Ex3-log.pdf}
	\caption{Double logarithmic plot of the generated data $y = f(x)$ with 10000 samples.}
  \label{fig:ex3-log}
\end{figure}



\section{Trajectory Of A Random Walk}
\label{sec:Trajectory Of A Random Walk}

\subsection{Normal Walker}
\label{sec:Normal Walker}
In this part we take a look at the normal walker, i.e. a walker that can return.
In \autoref{fig:dist-returning} the trace for eight returning walkers is plotted. 
\begin{figure}[H]
	\centering
	\includegraphics{build/many_walks.pdf}
	\caption{Eight random walker`s trace plotted for 1000 steps.}
  \label{fig:dist-returning}
\end{figure}


\end{document}
