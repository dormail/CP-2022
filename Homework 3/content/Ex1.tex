\section{Random deposition and diffusion model}
\label{sec:exercise-1}
For this part the random deposition and diffusion model has been implemented in python as it has been discussed in the
lecture. 

\subsection{Drawing a growing surface}
\label{sec:RDD-drawing}
Since the behavior of the surface is closely related to the size $L$, the images are provided with a proportionality
between $L$ and the number of particles deposited and diffused on it. In 
\autoref{fig:growing-surface-1},
\autoref{fig:growing-surface-10} and
\autoref{fig:growing-surface-100}
the surface is shown after $L$, $10L$ and $100L$ particles dropped with $L=50$.
\begin{figure}
	\centering
  \includegraphics[width=0.5\textwidth]{build/Ex1-draw-1.pdf}
  \caption{The growing surface after $L = 50$ iterations.}
  \label{fig:growing-surface-1}
\end{figure}
\begin{figure}
	\centering
  \includegraphics[width=0.5\textwidth]{build/Ex1-draw-10.pdf}
  \caption{The growing surface after $10L = 500$ iterations.}
  \label{fig:growing-surface-10}
\end{figure}
\begin{figure}
	\centering
  \includegraphics[width=0.5\textwidth]{build/Ex1-draw-100.pdf}
  \caption{The growing surface after $100L = 5000$ iterations.}
  \label{fig:growing-surface-100}
\end{figure}

\subsection{Roughness exponent}
To measure the roughness exponent $\alpha$ many surfaces are going to be simulated for different $L$. As it will be
shown later $W(L,t)$ has a large variance, so the procedure is as following:
\begin{itemize}
  \item Choose $L$
  \item Have $10$ surface instances growing for a large time $T \gg L$
  \item Measure $W(L,t)$ and take the average for the interval $t > t_x$. For simplicity we chose $t_x = L$.
\end{itemize}
The result of this procedure is a function $W_\text{ave}(L)$ which we can analyse. In particular we are looking for the
roughness exponent which we get by fitting the polynomial 
\[
  W(L) = A L^\alpha
\]
to the measured $W$. This gives us
\[
  \alpha = 0.187 \pm 0.014.
\]
The measured data and curve fit function are plotted in \autoref{fig:alpha}.


\begin{figure}
	\centering
  \includegraphics[width=0.5\textwidth]{build/Ex1-alpha.pdf}
  \caption{The surface`s roughness after a long time for different $L$.}
  \label{fig:alpha}
\end{figure}

\subsection{Growth exponent }
In this part the behavior for small $t$, get analysed. According to the exercise the roughness can be described by
\[
   W(L,t) = At^\beta.
\]
As it can be seen in \autoref{fig:instable-W}, the exact growth and width can depend on the run because of the random
nature of the model. 

\begin{figure}
	\centering
  \includegraphics[width=0.5\textwidth]{build/many_widths.pdf}
  \caption{Multiple runs of growing surface to show the instabilty of $W(L=50,t)$.}
  \label{fig:instable-W}
\end{figure}
Because of that, for the next analysis, $W(L,t)$ gets computed as an average over $100$ runs. Although the runtime is
suffering, the results are more stable.

For $L=T=50$ we get a curve fit result of
\[
  \beta = 0.353 \pm 0.007.
\]
To check wether this value is independent of $L$, the same analysis has been done with $L = T = 100$ with a fit result
of
\[
  \beta = 0.358 \pm 0.005,
\]
Which is in good aggrement with $\beta(L=T=50)$. The actual data of $W(L,t)$ and the curve fits are shown in
\autoref{fig:beta50} and
\autoref{fig:beta100}.
\begin{figure}
	\centering
  \includegraphics[width=0.5\textwidth]{build/beta_50.pdf}
  \caption{$W(L=50,t)$, averaged over 100 runs.}
  \label{fig:beta50}
\end{figure}
\begin{figure}
	\centering
  \includegraphics[width=0.5\textwidth]{build/beta_100.pdf}
  \caption{$W(L=100,t)$, averaged over 100 runs.}
  \label{fig:beta100}
\end{figure}
