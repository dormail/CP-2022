\section{Verification of Zipf`s law}
\label{sec:exercise-5}
In the following we will verify Zipf`s law, which describes the statistical behaviour of words in texts. It states that
the $n$-th most common word has $A/n$ appearences in the text. As a data set I used the four Homework sheets from this
course. To extract
the words I Ctrl-A Ctrl-C`ed the PDF of the sheets and pasted it in a text file. The line breaks with a dash (-) had to
be fixed manually. They have been imported in Python which counted the words after little cleaning (e.g. removing
linebreaks alltogether and removing dots, kommas). Afterwards I used the most appearing word`s occurences (which was
``a`` with $A = 1175$ occurences) and drew the curve according to this initial value. 

The result is shown in \autoref{fig:zipf} for the 50 most common words. 
While the law performs well for words in 10th place and above, it is not so reliable for 2nd until 5th place. Those
values can be quite unstable because the letter `a` is not only common in the english language but also hat appearences
in equations. Other letters which have a similar problem but do not appear as a single word (e.g. i, x, e) have been
excluded.
\begin{figure}
	\centering
  \includegraphics[width=0.5\textwidth]{build/Ex5.pdf}
  \caption{Plot of appearences and behaviour according to Zipf`s law.}
  \label{fig:zipf}
\end{figure}

