\section{Local structure of web pages}
\label{sec:exercise-3}
For the network analysis we use wikipedia articles. First some python functions scraping the links of articles has been
written. After naming a starting article has been named, the graph gets filled with $N$ nodes, each corresponding to a
article. Later a function is iterating through each nodes article and adds the graph edges.

A visualization is shown in \autoref{fig:Ex3-network}. By counting the edges we can also show the distribution of degree
for edges, shown in \autoref{fig:Ex3-edges}.
\begin{figure}
	\centering
  \includegraphics[width=0.5\textwidth]{build/Ex3-network_graph.pdf}
  \caption{Visualization of the network. Because it is computationally infeaseble to go deep since the article number
  grows exponentially the starting article is a well connected center.}
  \label{fig:Ex3-network}
\end{figure}
\begin{figure}
	\centering
  \includegraphics[width=0.5\textwidth]{build/Ex3-edges.pdf}
  \caption{Distribution of the amount of neighbours for articles.}
  \label{fig:Ex3-edges}
\end{figure}

