\section{Benford`s law}
\label{sec:exercise-4}
In this part we take a look at Benford`s law which suggests that for some numerical data set the initial digits $d$ are
distributed logarithmically
\begin{equation}
  p(d) = \log\left(1 + \frac{1}{d}\right)
  \label{eqn:benford}
\end{equation}
and is based on the assumption that the dataset is not limited to specific to a scale.
%
%
% The proof
%
%

One example for a dataset that obeys Benfords law are physical constants, which can be scraped from wikipedia as it has
been done in the programming tutorial. \autoref{fig:benford-decimal-1} shows the distribution for the first digit in the
dataset and shows good agreement between the real distribution and what Benfords law suggests. The scale free behaviour
has been tested as well by multiplying all numbers with $5$ before counting first digits in
\autoref{fig:benford-decimal-5}, and as it can be seen it still holds.

One really fancy feature of Benford`s law is that it can be applied to different number systems: By changing the base of
the logarithm in \autoref{eqn:benford}, it can be applied to every other system as well. \autoref{fig:benford-octal} and
\autoref{fig:benford-hexadec} show that it is still applicable for octal (base 8) and hexadecimal (base 16).
\begin{figure}
	\centering
  \includegraphics[width=0.5\textwidth]{build/Ex4-decimal.pdf}
  \caption{Benfords law for decimal system.}
  \label{fig:benford-decimal-1}
\end{figure}
\begin{figure}
	\centering
  \includegraphics[width=0.5\textwidth]{build/Ex4-decimal-factor5.pdf}
  \caption{Scale free behaviour of Benford`s law: The same distribution as in \autoref{fig:benford-decimal-1} with the
  dataset multiplied with $5$.}
  \label{fig:benford-decimal-5}
\end{figure}
\begin{figure}
	\centering
  \includegraphics[width=0.5\textwidth]{build/Ex4-octal.pdf}
  \caption{Benfords law for octal (base 8) number system.}
  \label{fig:benford-octal}
\end{figure}
\begin{figure}
	\centering
  \includegraphics[width=0.5\textwidth]{build/Ex4-hexadec.pdf}
  \caption{Benfords law for hexadecimal (base 16) number system.}
  \label{fig:benford-hexadec}
\end{figure}


