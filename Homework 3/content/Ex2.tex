\section{Generating fractal with probabilistic iteration}
\label{sec:exercise-2}

\subsection{Generating a fractal and rebuilding one}
I reproduced the fractal in \autoref{fig:fractal}.

\subsection{Measuring fractal dimension}
Luckily the fractals are embedded in a 2D-Surface so we can use the box counting method. With the box counting method,
the whole surface gets covered with $N$ boxes to approximate the surface of the fractal. In an iteration the boxes are
made smaller and $N$ gets raised. If you have a nice, well behaving two dimensional object, doubling $N$ (so splitting
boxes in half) results in approximately double the amount of boxes filled. With a fractal however, this can be any
number. In \autoref{fig:fractal-dimension} this can be seen for the given fractal.

\begin{figure}
	\centering
  \includegraphics[width=0.5\textwidth]{build/my_rebuild.png}
  \caption{The fractal as described in the exercise.}
  \label{fig:fractal}
\end{figure}
\begin{figure}
	\centering
  \includegraphics[width=0.5\textwidth]{build/fractal_dimension.png}
  \caption{The result of box counting for \autoref{fig:fractal}. The curve fit suggests a dimension of $\frac12$.}
  \label{fig:fractal-dimension}
\end{figure}
